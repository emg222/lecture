\documentclass[11pt]{article}
\usepackage[margin=1in]{geometry}

% Packages we need
\usepackage{amsmath}
\usepackage{amsfonts}
\usepackage{mathtools}
\usepackage{amsthm}
\usepackage{float}
\usepackage{graphicx}
\usepackage{listings}
\usepackage{color} %red, green, blue, yellow, cyan, magenta, black, white

% Header packages
\usepackage{fancyhdr}
\fancyhf{}
\pagestyle{fancy}

% Algorithms
\usepackage{algorithmic}
\usepackage{algorithm}

% Formatting document
\setcounter{secnumdepth}{0}
\setlength{\parindent}{0in}
\setlength{\parskip}{0.5em}

% MATLAB code
\definecolor{mygreen}{RGB}{28,172,0} % color values Red, Green, Blue
\definecolor{mylilas}{RGB}{170,55,241}
\lstset{language=Matlab,%
    %basicstyle=\color{red},
    breaklines=true,%
    morekeywords={matlab2tikz},
    keywordstyle=\color{blue},%
    morekeywords=[2]{1}, keywordstyle=[2]{\color{black}},
    identifierstyle=\color{black},%
    stringstyle=\color{mylilas},
    commentstyle=\color{mygreen},%
    showstringspaces=false,%without this there will be a symbol in the places where there is a space
    numbers=left,%
    numberstyle={\tiny \color{black}},% size of the numbers
    numbersep=9pt, % this defines how far the numbers are from the text
    emph=[1]{for,end,break},emphstyle=[1]\color{red}, %some words to emphasise
}

% Commands
\DeclarePairedDelimiter\ceil{\lceil}{\rceil}
\DeclarePairedDelimiter\floor{\lfloor}{\rfloor}
\newcommand{\ws}{\text{ }}
\newcommand{\e}[1]{\times 10^{#1}}

% Header
\lhead{\textsc{}} % TODO: enter title here
\rhead{\textsc{Eric Gao -- emg222}} % Authors
\setlength{\headheight}{0.5in}
\cfoot{\thepage}

% Title
\title{CS 5220 -- Sep. 08 Preclass Questions} %TODO: enter title here
\author{
  \begin{tabular}{l c l}
    Eric Gao & -- & emg222\\
  \end{tabular}\\
  \rule{\linewidth}{0.4pt}
}
\date{}


\begin{document}
    \thispagestyle{empty}
    \maketitle

    \section*{Question 1}
        The memory-based arithmetic intensity is defined as:
        \begin{align*}
            \text{AI} = \frac{\text{\# flops}}{\text{\# bytes transferred between memory and cache}}
        \end{align*}

        In the innermost loop, we have to perform 3 floating point operations on every iteration, which results in $3N$ flops for the inner loop. The total memory needed to perform the dot product between row $i$ of $A$ and column $j$ of $B$ is $2N\times 8\text{ bytes} = 16N\text{ bytes}$. Because this significantly larger than the size of our L3 cache, we can safely assume that under LRU, we need fetch all $16N$ bytes of data, everytime the innermost loop is executed. Therefore our memory-based AI is:
        \begin{align*}
            \text{AI} = \frac{3}{16}
        \end{align*}

    \section*{Question 2}
        As shown in the previous question, we need $16N$ bytes of data to perform the innermost loop. Therefore, we look at the next most inner loop. Top perform the first inner loop (for (j = 0; ... )), we need $8N^2 + 8N$ bytes of memory. We need every entry from $B$ which equals $8N^2$ bytes. We also need $8N$ bytes to keep row $i$ of $A$ around. To perform the first inner loop, we do $3N^2$ flops. The arithmetic intensity is now approximately:
        \begin{align*}
            \text{AI} = \frac{3}{8}
        \end{align*}

    \section*{Question 3}
        We need to keep all of $A$, $B$, and $C$ in cache, which equals $24N^2$ bytes. We then need to write to all of $C$ back to main memory, which results in another $8N^2$ memory operations. This gives a total of $32N^2$ bytes transferred between memory and cache. We then have a total of $3N^3$ flops. Therefore, our arithmetic intensity is:
        \begin{align*}
            \text{AI} = \frac{3N}{32}
        \end{align*}

    \section*{Question 4}
        The L1 cache has $2^{15}$ bytes. We can just solve for $N$:
        \begin{align*}
            24N^2 &\leq 2^{15}\\
            N &\leq \sqrt{1365.33} = 36.95
        \end{align*}

        $N$ must be a positive integer, so we choose $N$ to be 36. Because all of $A$, $B$ and $C$ can fit into the L3 cache, we can just use the formula obtained in Question 3 to obtain the AI. 
        \begin{align*}
            \text{AI}_{L1} = \frac{3(36)}{32} = 3.375
        \end{align*}

        The L2 cache has $2^{18}$ bytes. Solving again for $N$:
        \begin{align*}
            24N^2 &\leq 2^{18}\\
            N &\leq \sqrt{10922.66} = 104.5
        \end{align*}

        We therefore choose $N$ to be 104. The AI is then given by:
        \begin{align*}
            \text{AI}_{L2} = \frac{3(104)}{32} = 9.75
        \end{align*}

        The L3 cache has on the order of 6 million bytes:
        \begin{align*}
            24N^2 &\leq 6000000\\
            N &\leq \sqrt{250000} = 500
        \end{align*}

        We therefore choose $N$ to be 500. The AI is therefore:
        \begin{align*}
            \text{AI}_{L3} = \frac{3(500)}{32} = 46.875
        \end{align*}

    \section*{Question 5}
        We first calculate the FLOPs/s on the CPU:
        \begin{align*}
            2 \frac{\text{flops}}{\text{FMA}} \times 8 \frac{\text{FMA}}{\text{cycle}}\times (2.4\times 10^9)\frac{\text{cycles}}{\text{second}} \times 4\text{ cores} = 153.6 \text{ GFLOPs / s}
        \end{align*}
        
        To calculate arithmetic intensity, we divide FLOP/s by the memory bandwidth:
        \begin{align*}
            \text{AI} = \frac{153.6 \text{ GFLOPs / s}}{25.6 \text{ GB / s}} = 6 \text{ FLOPs / byte}
        \end{align*}

    \section*{Question 6}
        We can solve for $N$:
        \begin{align*}
            6 &= \frac{3N}{32}\\[0.5em]
            64 &= N
        \end{align*}

        When $64 \leq N \leq 500$, the naive matmul is CPU-bound.

    \section*{Question 7}
        When $64 \leq N \leq 500$, the naive matmul is CPU-bound. Otherwise, the operation is memory bound. So when $N < 64$, the plot of Flops/s will be linearly increasing. The slope of this linear increase is:
        \begin{align*}
            slope = \frac{3(25.6\times 10^9)}{32} = 2.4 \times 10^9
        \end{align*}

        This will be linearly increasing until we hit 153.6 Flops / s at $N = 64$. This line will then be straight until $N = 500$. The naive matmul will become memory-bound at this point. The Flop/s will decrease slowly at first, faster when the cache is smaller than $8N^2$, and faster still when the cache is smaller than $16N$.

\end{document}